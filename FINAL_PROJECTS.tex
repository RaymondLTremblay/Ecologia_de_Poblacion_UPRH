% Options for packages loaded elsewhere
\PassOptionsToPackage{unicode}{hyperref}
\PassOptionsToPackage{hyphens}{url}
%
\documentclass[
]{article}
\usepackage{amsmath,amssymb}
\usepackage{lmodern}
\usepackage{iftex}
\ifPDFTeX
  \usepackage[T1]{fontenc}
  \usepackage[utf8]{inputenc}
  \usepackage{textcomp} % provide euro and other symbols
\else % if luatex or xetex
  \usepackage{unicode-math}
  \defaultfontfeatures{Scale=MatchLowercase}
  \defaultfontfeatures[\rmfamily]{Ligatures=TeX,Scale=1}
\fi
% Use upquote if available, for straight quotes in verbatim environments
\IfFileExists{upquote.sty}{\usepackage{upquote}}{}
\IfFileExists{microtype.sty}{% use microtype if available
  \usepackage[]{microtype}
  \UseMicrotypeSet[protrusion]{basicmath} % disable protrusion for tt fonts
}{}
\makeatletter
\@ifundefined{KOMAClassName}{% if non-KOMA class
  \IfFileExists{parskip.sty}{%
    \usepackage{parskip}
  }{% else
    \setlength{\parindent}{0pt}
    \setlength{\parskip}{6pt plus 2pt minus 1pt}}
}{% if KOMA class
  \KOMAoptions{parskip=half}}
\makeatother
\usepackage{xcolor}
\usepackage[margin=1in]{geometry}
\usepackage{graphicx}
\makeatletter
\def\maxwidth{\ifdim\Gin@nat@width>\linewidth\linewidth\else\Gin@nat@width\fi}
\def\maxheight{\ifdim\Gin@nat@height>\textheight\textheight\else\Gin@nat@height\fi}
\makeatother
% Scale images if necessary, so that they will not overflow the page
% margins by default, and it is still possible to overwrite the defaults
% using explicit options in \includegraphics[width, height, ...]{}
\setkeys{Gin}{width=\maxwidth,height=\maxheight,keepaspectratio}
% Set default figure placement to htbp
\makeatletter
\def\fps@figure{htbp}
\makeatother
\setlength{\emergencystretch}{3em} % prevent overfull lines
\providecommand{\tightlist}{%
  \setlength{\itemsep}{0pt}\setlength{\parskip}{0pt}}
\setcounter{secnumdepth}{-\maxdimen} % remove section numbering
\ifLuaTeX
  \usepackage{selnolig}  % disable illegal ligatures
\fi
\IfFileExists{bookmark.sty}{\usepackage{bookmark}}{\usepackage{hyperref}}
\IfFileExists{xurl.sty}{\usepackage{xurl}}{} % add URL line breaks if available
\urlstyle{same} % disable monospaced font for URLs
\hypersetup{
  pdftitle={Final projects!},
  pdfauthor={BIOL4558},
  hidelinks,
  pdfcreator={LaTeX via pandoc}}

\title{Final projects!}
\author{BIOL4558}
\date{Noviembre 2021}

\begin{document}
\maketitle

{
\setcounter{tocdepth}{2}
\tableofcontents
}
\hypertarget{proyecto-final-biol-4558}{%
\subsection{Proyecto Final BIOL 4558}\label{proyecto-final-biol-4558}}

Los estudiantes trabajarán en grupos de \textasciitilde{} 2-3 personas
para realizar un análisis de viabilidad de la población (PVA) para
clasificar las acciones de conservación o manejo para una especie de
interés de conservación (¡especie de su elección!). La calificación se
basará en los productos terminados (presentaciones escritas y orales),
así como en la participación y evaluaciones de pares.

\hypertarget{puntos-importante}{%
\subsection{Puntos importante}\label{puntos-importante}}

\begin{itemize}
\item
  Seleccionar una especie y una pregunta de interés.
\item
  Realice una revisión exhaustiva de la literatura sobre su especie de
  interés; recopile toda la información que pueda sobre el historial de
  vida, las tasas vitales clave y su variabilidad.
\item
  Construir un modelo PVA, parametrizado utilizando la mejor información
  disponible de la literatura (plataformas de software PVA gratuitas:
  InsightMaker o R).
\item
  Utilice su modelo PVA para abordar una pregunta de conservación o
  manejo y \emph{escriba los resultados}.
\item
  ¡Presenta tus resultados a la clase!
\end{itemize}

\hypertarget{cronologuxeda}{%
\subsection{Cronología}\label{cronologuxeda}}

\begin{itemize}
\tightlist
\item
  \textbf{PROPUESTA: } ¡Vence el Lunes 15 de noviembre!
\item
  \textbf{PVA INICIAL } Vence el Lunes 22 de noviembre: descripción de
  cómo funciona su modelo PVA y de dónde provienen las estimaciones de
  los parámetros.
\item
  ** ENVIAR EL BORRADOR PARA REVISIÓN POR PARES ** Vence el 24 de
  noviembre a las 5 p.m. (para revisión por pares el 26 de noviembre)
\item
  \textbf{BORRADOR DEL DOCUMENTO FINAL: } Se vence el miércoles 1 de
  diciembre a las 5 p.m.
\item
  \textbf{DOCUMENTO FINAL FINAL: } ¡El ultimo dia de clase!
\item
  \textbf{PRESENTACIONES FINALES: } ¡Los dos últimos días de las clases!
\end{itemize}

\hypertarget{primera-asignaciuxf3n-relacionada-con-el-proyecto-la-propuesta}{%
\subsection{Primera asignación relacionada con el proyecto: ¡la
propuesta!}\label{primera-asignaciuxf3n-relacionada-con-el-proyecto-la-propuesta}}

Su propuesta de proyecto (aproximadamente 1 pg) consistirá en:

\begin{enumerate}
\def\labelenumi{\arabic{enumi}.}
\tightlist
\item
  \textbf{Título }: debe indicar qué especie está modelando y al menos
  insinuar qué pregunta se abordará
\item
  \textbf{Participantes del proyecto }: proporcione los nombres de todos
  los participantes (2-3).
\item
  \textbf{Pregunta (s) de investigación }: proporcione una o más
  preguntas comprobables y relevantes para la gestión que planea abordar
  utilizando su modelo de población.
\item
  \textbf{Justificación del proyecto }: Describa la justificación para
  elegir esta especie y la cuestión de gestión/conservación. ¡Sea lo más
  específico posible! ¡Convence a tus instructores de que tu proyecto es
  ** interesante **!
\item
  \textbf{Fuentes de datos }: Describa una o más fuentes de datos /
  información que planea usar para guiar la construcción y
  parametrización de PVA (¡queremos asegurarnos de que su proyecto sea
  \textbf{posible }!)
\end{enumerate}

\textbf{NOTA sobre las fuentes de datos:} - Los parámetros más
importantes que necesita para un modelo PVA son \emph{tasas de
supervivencia } (generalmente estructuradas por edad), \emph{tasas de
fecundidad } (a menudo también estructuradas por edad),
\emph{estocasticidad } (cantidad de variabilidad aleatoria en esas tasas
vitales), \emph{abundancia inicial }, \emph{tasas de dispersión } y
\emph{capacidad de carga (K) }. - Ya hemos repasado muchos de estos
conceptos en clase: ¡PVA es solo una forma de unir todos estos conceptos
en un solo modelo! Puede obtener estas estimaciones de parámetros
directamente de la literatura publicada (o informes no publicados), o
puede estimar estos parámetros usted mismo utilizando datos sin procesar
(que aún no hemos revisado en clase). Si todo lo demás falla, puede usar
registros publicados para especies similares (asumiendo que las tasas
vitales son aproximadamente las mismas).

\hypertarget{segunda-asignaciuxf3n-relacionada-con-el-proyecto-el-pva-inicial}{%
\subsection{Segunda asignación relacionada con el proyecto: ¡el PVA
inicial!}\label{segunda-asignaciuxf3n-relacionada-con-el-proyecto-el-pva-inicial}}

Su PVA inicial parametrizado consistirá en:

\begin{enumerate}
\def\labelenumi{\arabic{enumi}.}
\item
  Un modelo PVA listo para ejecutar, en cualquier marco de software que
  esté utilizando (InsightMaker, R). Este modelo no necesita ser el
  modelo ``final'', pero debería funcionar (por ejemplo, producir
  resultados) y las estimaciones de los parámetros deberían ser
  razonables y estar respaldadas por pruebas (ver más abajo).
\item
  Una justificación/evidencia escrita para cada una de sus estimaciones
  de parámetros y otras decisiones clave de modelado.

  \begin{enumerate}
  \def\labelenumii{\Alph{enumii}.}
  \tightlist
  \item
    Para cada decisión importante que tomó al construir el modelo,
    describa la información que utilizó para respaldar su decisión (y
    la(s) fuente(s) de esta información). Las decisiones incluyen:
    cuántas poblaciones, abundancia(s) inicial(es), cuántas etapas, cómo
    opera la dependencia de la densidad para su especie (por ejemplo,
    qué tasas vitales dependen de la densidad), efectos Allee, tasas de
    supervivencia y fecundidad, estocasticidad ambiental, catástrofes,
    etc. Cuando sea apropiado, siéntase libre de incrustar figuras,
    mapas, etc. de sus fuentes de información.
  \end{enumerate}

  B. ¿Qué escenarios planea probar (por ejemplo, escenarios de manejo,
  cambio climático, cosecha, etc.) y cómo estos escenarios lo ayudarán a
  abordar sus preguntas de investigación?

  C. Literatura citada (el formato no es importante, siempre que incluya
  toda la información clave).
\end{enumerate}

\hypertarget{tercera-tarea-relacionada-con-el-proyecto-el-proyecto-escrito-borrador}{%
\subsection{Tercera tarea relacionada con el proyecto: ¡el proyecto
escrito
(borrador)!}\label{tercera-tarea-relacionada-con-el-proyecto-el-proyecto-escrito-borrador}}

Se espera que su trabajo final escrito tenga el estilo de un
\textbf{manuscrito científico }. Es decir, debe tener secciones de
\emph{introducción }, \emph{métodos }, \emph{resultados } y
\emph{discusión }, y debe citar la literatura relevante.

\textbf{No hay pautas de formato ni límites de página }. Sea lo más
conciso posible sin dejar de cubrir todos los elementos clave de la
rúbrica. Puede utilizar cualquier formato estándar para sus referencias,
¡todo lo que le pedimos es que sea coherente!

\hypertarget{introducciuxf3n}{%
\subsubsection{Introducción}\label{introducciuxf3n}}

Aquí es donde presenta el tema y describe por qué es importante. También
debe incluir aquí sus preguntas de investigación, junto con cualquier
hipótesis que esté probando. ¡Puede reciclar y dar cuerpo al material de
su propuesta! Debe tener al menos 2-3 párrafos.

\hypertarget{muxe9todos}{%
\subsubsection{Métodos}\label{muxe9todos}}

Aquí es donde usted describe su PVA con suficiente detalle como para que
\emph{¡otro investigador de vida silvestre lo pueda replicar }!Puede
reciclar gran parte de su asignación de ``PVA inicial'' aquí.

Recuerde justificar todas sus decisiones en términos de parametrización.

Además, debe describir cómo abordó las preguntas de investigación que
presentó en la sección de introducción (arriba)

\begin{itemize}
\item
  ¿Qué diferentes escenarios corriste?
\item
  ¿Cómo utilizó los resultados de la simulación para abordar sus
  preguntas?

  \begin{itemize}
  \tightlist
  \item
    ¿Cómo visualizó los resultados?
  \item
    ¿Realizó algún análisis estadístico?
  \end{itemize}
\end{itemize}

\hypertarget{resultados}{%
\subsubsection{Resultados}\label{resultados}}

Aquí es donde describe y presenta los resultados relevantes de su modelo
(la mayoría se puede reciclar de su asignación de `PVA final').
Proporcione figuras y tablas para resumir sus resultados
\emph{relevantes }. Debe incluir al menos 2-3 cifras como parte de su
sección de resultados. Cada figura debe tener una leyenda informativa
colocada directamente debajo de la figura.

Con PVA puede ser difícil decidir qué presentar y qué no presentar.
Pregúntese siempre: ¿es * relevante * para sus preguntas principales??

\hypertarget{discusiuxf3n}{%
\subsubsection{Discusión}\label{discusiuxf3n}}

Aquí es donde usted describe lo que realmente nos dicen sus resultados
con respecto a sus preguntas principales.

Además, en su discusión debe incluir:

\begin{itemize}
\tightlist
\item
  Describa cualquier pregunta nueva que surgió en el proceso de
  construcción y ejecución de su PVA.
\item
  Describe cualquier defecto potencial de tu PVA y cómo podrías
  mejorarlo en el futuro.
\item
  Describa cualquier investigación futura que pueda ser útil para
  abordar sus principales preguntas.
\end{itemize}

Se puede encontrar una rúbrica para la presentación escrita final.
\href{WrittenProject_rubric.docx}{here}.

\hypertarget{cuarta-tarea-relacionada-con-el-proyecto-la-presentaciuxf3n-oral}{%
\subsection{Cuarta tarea relacionada con el proyecto: ¡la presentación
oral!}\label{cuarta-tarea-relacionada-con-el-proyecto-la-presentaciuxf3n-oral}}

La presentación de su grupo debe tener el estilo de una presentación
oral en una conferencia. ¡Consulte a continuación para obtener
orientación sobre cómo dar una buena presentación oral!

Las presentaciones se llevarán a cabo durante nuestros dos últimos
períodos de laboratorio.

Tendrá 15 minutos para realizar sus presentaciones, con al menos 3
minutos adicionales para preguntas.

Le entregaremos un formulario simple de revisión por pares para que
pueda dar su opinión a sus compañeros.

\hypertarget{se-puede-encontrar-una-ruxfabrica-para-las-presentaciones.-here}{%
\section{\texorpdfstring{Se puede encontrar una rúbrica para las
presentaciones.
\href{presentationProject_rubric.docx}{here}}{Se puede encontrar una rúbrica para las presentaciones. here}}\label{se-puede-encontrar-una-ruxfabrica-para-las-presentaciones.-here}}

\hypertarget{consejos-para-una-gran-presentaciuxf3n}{%
\subsubsection{¡Consejos para una gran
presentación!}\label{consejos-para-una-gran-presentaciuxf3n}}

Aquí hay algunas notas generales sobre cómo preparar y realizar una gran
presentación: - Para todas las diapositivas: \emph{¡Menos palabras, más
imágenes }! Las diapositivas llenas de palabras a menudo hacen que las
personas dejen de prestar atención. Todos ustedes están trabajando con
especies carismáticas, por lo que no deberían tener problemas para
encontrar buenas imágenes que capten la atención de las personas. En
cuanto a las palabras, ¡3-5 puntos como máximo! NO se necesitan
oraciones completas en una presentación. Utilice una fuente como por
ejemplo \emph{sans-serif} grande para mejorar la legibilidad.

Nota: asegúrese de dar el crédito adecuado a todas las imágenes (por
ejemplo, ¿de qué sitio web proviene?)

\begin{itemize}
\item
  La emoción es contagiosa: asegúrese de transmitir su entusiasmo por el
  proyecto. Todos trabajaron muy duro y deberían estar orgullosos de lo
  que pudieron hacer. ¡Ahora puede compartir esta emoción con sus
  colegas! Muchos presentadores sin experiencia cometen el error de
  destacar solo lo que no pudieron lograr. Trate de evitar este escollo;
  las deficiencias deben limitarse a una sola diapositiva en la sección
  de discusión de la presentación.
\item
  Una buena presentación debe \emph{contar una historia }. Tiene un
  principio, un medio y un final. Cada diapositiva fluye lógicamente de
  lo que vino antes.
\item
  Hágalo simple, no use jerga, acrónimos, etc. excesivos. Buenas
  habilidades de comunicación significan transmitir sus puntos de la
  manera más simple y clara posible.
\item
  Cada presentación es una actuación. ¡El rendimiento requiere práctica!
  El trabajo real viene antes de la actuación: armar las diapositivas y
  practicar la presentación. Cuando llegue el momento de dar la
  actuación, relájese y diviértase, ¡el trabajo duro ha terminado!
\end{itemize}

Nota: no utilice fondos elegantes, mantenga las diapositivas lo más
limpias posible. Las animaciones pueden ser útiles, ¡pero la animación
excesiva es solo una distracción!

\hypertarget{diapositiva-de-tuxedtulo}{%
\paragraph{Diapositiva de título}\label{diapositiva-de-tuxedtulo}}

\begin{itemize}
\tightlist
\item
  ¡Haz un título interesante!
\item
  Incluya los nombres y afiliaciones de todos los coautores.
\end{itemize}

\hypertarget{introducciuxf3n-1}{%
\paragraph{Introducción}\label{introducciuxf3n-1}}

\begin{itemize}
\item
  ¡Dile a tu audiencia por qué debería importarles! Todos ustedes están
  trabajando con especies interesantes y todos están haciendo preguntas
  interesantes. Toda su audiencia está interesada en la conservación y
  el manejo de la vida silvestre. ¡No hay absolutamente ninguna razón
  por la que no puedas convencer a tu audiencia de que se preocupe por
  lo que tienes que decir!
\item
  Cite investigaciones anteriores al configurar su pregunta de
  investigación. Convéncenos de que su investigación llena un vacío de
  conocimiento clave. ¿Cuál es el contexto general de su investigación?
\item
  Una vez que haya enmarcado el problema, \emph{¡establezca claramente
  sus preguntas de investigación }!
\end{itemize}

\hypertarget{muxe9todos-1}{%
\paragraph{Métodos}\label{muxe9todos-1}}

\begin{itemize}
\item
  Proporcione un resumen muy abreviado de su modelo PVA. ¡No tiene
  tiempo para proporcionar todos los detalles que se encuentran en su
  sección de Métodos escrita! Así que intente destilar los elementos
  clave de sus métodos. ¡Tenga cuidado de no dedicar demasiado tiempo a
  esta sección! Los detalles que proporcione aquí deben ser claramente
  relevantes para sus preguntas de investigación. Todo lo que necesita
  hacer es convencer a sus compañeros de que sus métodos fueron
  apropiados para abordar sus preguntas.
\item
  Puede incluir diapositivas complementarias que brinden más detalles
  sobre sus métodos (después de la última diapositiva ``real''). ¡Puede
  ser útil consultar las diapositivas complementarias si alguien hace
  una pregunta sobre sus métodos!
\end{itemize}

\hypertarget{resultados-1}{%
\paragraph{Resultados}\label{resultados-1}}

\begin{itemize}
\item
  Incluir graficos claves que se relacionen con las preguntas
  principales.
\item
  ¡Para todos los gráficos, asegúrese de explicar qué representan los
  ejes! Además, asegúrese de que las etiquetas de los ejes sean lo
  suficientemente grandes para que se puedan leer desde la parte
  posterior de la sala.
\end{itemize}

\hypertarget{discusiuxf3n-1}{%
\paragraph{Discusión}\label{discusiuxf3n-1}}

\begin{itemize}
\item
  Resuma sus hallazgos clave: ¿qué aprendió?
\item
  Implicaciones de manejo: ¿cómo podrían los administradores de vida
  silvestre actuar sobre sus hallazgos?
\item
  Solo una diapositiva que analiza las deficiencias de su proyecto y
  cómo mejoraría su modelo y diseño si tuviera tiempo adicional.
\item
  Describa lo que usted u otros investigadores podrían hacer para
  ampliar sus hallazgos. ¿Cuáles son los próximos pasos?
\end{itemize}

\hypertarget{agradecimientos}{%
\paragraph{Agradecimientos}\label{agradecimientos}}

\begin{itemize}
\tightlist
\item
  Su diapositiva final debe reconocer a las personas que lo ayudaron con
  su proyecto pero que no figuran como coautores.
\end{itemize}

\hypertarget{enlaces-potencialmente-uxfatiles}{%
\subsection{Enlaces potencialmente
útiles!}\label{enlaces-potencialmente-uxfatiles}}

Conjuntos de datos disponibles públicamente, potencialmente para el
proyecto final \ldots{} (muchos enlaces cortesía de Tom Langen)

\hypertarget{biodiversity-data-clearinghouses-archives}{%
\subsubsection{BIODIVERSITY DATA CLEARINGHOUSES
/ARCHIVES}\label{biodiversity-data-clearinghouses-archives}}

\href{http://www.iucnredlist.org/}{International Union for the
Conservation of Nature (IUCN) Redlist} (Searchable list of the world's
threatened and endangered plants and animal species on the IUCN
Redlist.)

\href{http://www.biodiversityhotspots.org/Pages/default.aspx}{Conservation
International Global Biodiversity Hotspots} (Detailed data on the
attributes and threats to the world's global biodiversity hotspots.)

\href{http://www.nbii.gov/}{National Biological Information
Infrastructure} (Data archive and clearinghouse for biological data from
the US. Also provides standards for metadata.)

\href{http://www.ice.ucdavis.edu/bioinventory/bioinventory.html}{Biological
Inventories of the World's Protected Areas} (Searchable species
occurrence records and species lists for over 1,400 protected areas
around the globe.)

\href{http://data.gbif.org/welcome.htm}{Global Biodiversity Information
Facility} (An enormous clearinghouse of biodiversity data)

\href{https://migbirdapps.fws.gov/mbdc/databases/db_selection.html}{USGS
avian data portal}

\href{http://www3.imperial.ac.uk/cpb/research/patternsandprocesses/gpdd}{Global
Population Dynamics Data Base(GPDD)} (5000 population size time series
for 1400 species, most of which have at least ten years of data. There
are data on the natural history of the organism and the location \&
method of sampling.)

\href{https://www.imperial.ac.uk/cpb/gpdd2/secure/login.aspx}{GPDD,
alternative link}

\href{http://www.pwrc.usgs.gov/BBS/}{USGS Breeding Bird Survey}
(Breeding bird survey data back to 1966)

\href{http://www.pwrc.usgs.gov/point/}{Bird Point Count Database}
(Depository of bird point-count data from across the US.)

\href{http://www.bsc-eoc.org/birdmon/default/main.jsp}{Bird Studies
Canada Nature Counts} (Bird survey data archive for Canada, includes
point counts and many other types of surveys.)

\href{http://www.avianknowledge.net/content/datasets}{Avian Knowledge
Network} (Archive of aggregated bird surveys from many organizations and
studies across throughout the western hemisphere, including Latin
America.)

\href{http://www.natureserve.org/getData/index.jsp}{NatureServe} (Data
on species of plants and animals in the Western Hemisphere, including
detailed range maps)

\href{https://my.usgs.gov/bpd/}{USGS bat data portal}

\href{https://compadredb.wordpress.com/2015/10/05/introducing-the-comadre-animal-matrix-database/}{Comadre
and Compadre matrix demography database}

\hypertarget{government-agency-data-portals}{%
\subsubsection{GOVERNMENT AGENCY DATA
PORTALS}\label{government-agency-data-portals}}

\href{http://www.nationalatlas.gov/}{National Atlas} (Geospatial data on
the environment, economy, and people of the US).

\href{http://www.agcensus.usda.gov/}{US Department of Agriculture Census
of Agricultural Data} (Authoritative data on all aspects of agriculture
in the US.)

\href{http://www.cdc.gov/datastatistics/}{Centers for Disease Control \&
Prevention Data \& Statistics} (Comprehensive data on all aspects of
disease epidemiology.)

\href{http://waterdata.usgs.gov/nwis}{USGS Water Data for the Nation}
(Hydrological and water-quality data from across the US.)

\href{http://diseasemaps.usgs.gov/index.html}{USGS Survey Disease Maps}
(US County-scale maps of incidence patterns of various mosquito-vectored
diseases)

\href{http://www.mrlc.gov/}{The Multi-resolution Land Characteristics
Consortium (MRLC) National Land Cover Database} (Land cover or land use,
canopy cover, and impermeable surface area of the entire US, at a
resolution of 30 m x 30 m, based on remote sensing data from satellite
imagery.)

\href{http://www.fws.gov/wetlands/}{US Fish \& Wildlife Service National
Wetlands Inventory} (Wetlands greater than 1 acre are mapped and
classified throughout the US, Puerto Rico and US territories. Data can
be examined using the
\href{http://www.fws.gov/wetlands/Data/Mapper.html}{Wetland Mapper} and
then downloaded for use by a GIS application, or can by inspected
directly using
\href{http://www.fws.gov/wetlands/Data/GoogleEarth.html}{Google Earth})

\href{http://fia.fs.fed.us/}{USDA Forest Inventory and Analysis National
Program} \href{http://fia.fs.fed.us/tools-data/default.asp}{Forest
Inventory Data Online (FIDO)} (Highly-detailed periodic surveys of
forest composition at sites throughout the US.)

\href{http://www.usgs.gov/}{US Geological Survey} (Reports, data
analysis, maps, and raw data on a diversity of topics related to
environmental science, including biodiversity and emerging diseases.)

\href{http://www.ncdc.noaa.gov/oa/ncdc.html}{NOAA National Climate Data
Center} (Extensive data archives of climate data, including
paleoclimate.)

\hypertarget{environmental-data-clearinghouses}{%
\subsubsection{ENVIRONMENTAL DATA
CLEARINGHOUSES}\label{environmental-data-clearinghouses}}

\href{http://www.ecotrends.info/EcoTrends/}{Ecotrends} (Data archive and
data visualization tools for ecological data at sites distributed around
the US.)

\href{http://gcmd.nasa.gov/index.html}{NASA Global Change Master
Directory} (Data on all aspects of global change, includes data on
climate, land use, biodiversity and human dimensions.)

\href{http://daac.ornl.gov/index.shtml}{Oak Ridge National Laboratory
Distributed Active Archive Center for Biogeochemical Dynamics(ORNL
DAAC)} (A NASA-sponsored source for biogeochemical and ecological data
and models useful in environmental research.)

\href{http://www.p2erls.net/}{Pole to Pole Ecological Research Lattice
of Sites (P2ERLS)} (Portal to research stations and research networks,
including their data archives.)

\href{http://weatherspark.com/}{Weatherspark} (Visualized time-series
data on local climate at sites around the globe.)

\href{http://www.lternet.edu/}{Long Term Ecological Research (LTER)
Network} (Network of research stations that have standardized monitoring
programs as well as site-specific research. Sites are mandated to make
data publicly available on the web.)

\hypertarget{research-project-data-archives}{%
\subsubsection{RESEARCH PROJECT DATA
ARCHIVES}\label{research-project-data-archives}}

\href{http://datadryad.org/}{Dryad} (Data archives for bioscience data
from peer-reviewed journal articles from a large consortium of journals)

\href{http://esapubs.org/archive/archive_D.htm}{Ecological Society of
America (ESA) Data Registry} Archive of ecological and environmental
data from ESA publications)

\href{http://knb.ecoinformatics.org/knb/style/skins/nceas/index.jsp}{National
Center for Ecological Assessment \& Synthesis (NCEAS) Data Repository}
(Data archive of contributed data sets of all types of ecological data.)

\href{http://www.nceas.ucsb.edu/scicomp/}{NCEAS Scientific Computing
Database} (Clearinghouse of climatological, geospatial, and other data.
Also has shareware software for analysis.)

\end{document}
